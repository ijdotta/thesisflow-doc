% !TeX program = pdflatex
\documentclass[runningheads]{llncs}

% --- Packages ---
\usepackage[T1]{fontenc}
\usepackage[utf8]{inputenc}
\usepackage{graphicx}
\usepackage{hyperref}
\usepackage{booktabs}
\usepackage{csquotes}
\usepackage{amsmath}
\usepackage{enumitem}
\usepackage{listings}
\usepackage{xcolor}
\usepackage{caption}
\usepackage{subcaption}

% --- Metadata ---
\title{ThesisFlow: Plataforma Web para la Gestión y Visualización de Trabajos Finales en el DCIC}
\titlerunning{ThesisFlow}
\author{Ignacio Joaqu\'in Dotta}
\authorrunning{I. J. Dotta}
\institute{Departamento de Ciencias e Ingenier\'ia de la Computaci\'on (DCIC)\\
Universidad Nacional del Sur, Argentina\\
\email{ij.dotta@gmail.com}}

\begin{document}
\maketitle

\begin{abstract}
Este trabajo presenta \emph{ThesisFlow}, una plataforma web para la gestión y visualización de Trabajos Finales y Tesis de Grado del DCIC. El sistema integra en una única solución: (i) los flujos administrativos de carga y curado de información; (ii) un módulo de auto\-gestión para profesores; y (iii) un portal público de analíticas interactivas. Se describe la motivación y el contexto institucional, se analizan alternativas de solución, se especifican los objetivos perseguidos y se detalla la implementación a nivel de arquitectura, diseño e interfaces. A partir de la instancia desplegada se discuten usos concretos de la aplicación y se extraen conclusiones sobre su impacto en la transparencia y la toma de decisiones basada en datos. Finalmente, se identifican líneas de trabajo futuro para extender la plataforma e integrarla con otros sistemas institucionales.
\keywords{Sistemas de información académica \and Visualización de datos \and Ingeniería de software \and Aplicaciones web}
\end{abstract}

% --- 1. Motivación ---
\section{Motivaci\'on}
La gestión de Trabajos Finales y Tesis de Grado en el Departamento de Ciencias e Ingeniería de la Computación (DCIC) involucra a múltiples actores: secretaría, docentes, estudiantes y comunidad académica en general. Históricamente, la información asociada a estos proyectos se ha gestionado mediante herramientas heterogéneas (planillas, documentos aislados, sistemas parciales), lo que dificulta:

\begin{itemize}
  \item la trazabilidad del ciclo de vida de cada trabajo;
  \item el análisis de tendencias temáticas y de direcciones;
  \item la obtención de estadísticas confiables para la planificación académica; y
  \item la transparencia hacia la comunidad sobre la producción del departamento.
\end{itemize}

En este contexto, \emph{ThesisFlow} surge como una propuesta para centralizar la información, modernizar los procesos administrativos y habilitar un acceso más simple y estructurado a los datos históricos de Trabajos Finales y Tesis, tanto para usuarios internos como externos al DCIC.

[TODO: Agregar métricas o ejemplos concretos del estado previo, si se dispone de ellos.]

% --- 2. Exploración de soluciones ---
\section{Exploraci\'on de soluciones}
Previo al diseño de la solución propuesta se realizó una exploración de sistemas y enfoques ya existentes para la gestión académica y la visualización de datos:

\begin{itemize}
  \item Sistemas institucionales de gestión académica (por ejemplo, plataformas administrativas y repositorios digitales de tesis).
  \item Plataformas de e-learning y gestión de cursos (como Moodle), con módulos adicionales para trabajos finales.
  \item Soluciones ad-hoc basadas en hojas de cálculo y tableros de visualización externos.
  \item Portales públicos de tesis de otras universidades, que ponen el foco principalmente en la difusión, con capacidades limitadas de gestión interna.
\end{itemize}

En general, se observó que estas alternativas cubren parcialmente el problema: o bien se centran en la gestión interna pero sin visualizaciones públicas integradas, o se orientan a la difusión sin contemplar los flujos administrativos específicos del DCIC.

[TODO: Referenciar sistemas concretos y, en lo posible, incluir citas bibliográficas o institucionales.]

% --- 3. Alternativas ---
\section{Alternativas}
A partir de la exploración previa se analizaron distintas alternativas de solución:

\begin{enumerate}
  \item \textbf{Mantener el esquema actual con herramientas genéricas.}  
  Esto incluiría continuar utilizando planillas y documentos, complementados con tableros manuales. Si bien la inversión inicial es baja, los costos de mantenimiento y la propensión a errores humanos se mantienen elevados.

  \item \textbf{Adoptar una plataforma institucional o de terceros.}  
  Adaptar un sistema existente podría reducir esfuerzo de desarrollo, pero suele implicar restricciones fuertes en el modelo de datos y en la forma de exponer estadísticas específicas del DCIC.

  \item \textbf{Desarrollar un sistema propio, modular y extensible.}  
  Esta opción requiere mayor esfuerzo inicial, pero permite diseñar una solución alineada con las necesidades del departamento, controlando la evolución del sistema y su integración con otras herramientas.
\end{enumerate}

El proyecto se orienta por la tercera alternativa, priorizando la alineación con los requerimientos locales, la transparencia en el tratamiento de datos y la posibilidad de extender el sistema en futuras iteraciones.

% --- 4. Especificación de objetivos ---
\section{Especificaci\'on de objetivos}

\subsection{Objetivo general}
Diseñar e implementar una plataforma web integral para la gestión, el seguimiento y la visualización de Trabajos Finales y Tesis de Grado del DCIC, que sirva como fuente confiable de información para la gestión interna, los docentes y la comunidad académica en general.

\subsection{Objetivos específicos}
\begin{itemize}
  \item Centralizar la información de proyectos en una base de datos estructurada y consistente.
  \item Proveer interfaces de administración para la carga, edición y curado de Trabajos Finales y Tesis.
  \item Ofrecer un módulo de auto-gestión para profesores, acotado a los proyectos bajo su dirección.
  \item Desarrollar un portal público de analíticas interactivas sobre la producción académica del departamento.
  \item Incorporar mecanismos de respaldo y restauración de datos.
  \item Diseñar la arquitectura de forma modular y escalable, facilitando su evolución.
\end{itemize}

[TODO: Ajustar y completar los objetivos específicos de acuerdo al plan aprobado por el DCIC.]

% --- 5. Implementación ---
\section{Implementaci\'on}
En esta sección se describe la implementación de \emph{ThesisFlow} en términos de arquitectura, diseño y su alineación con los objetivos definidos.

\subsection{Arquitectura}
La solución adopta una arquitectura de tres capas, con una separación clara entre:

\begin{itemize}
  \item \textbf{Backend}: desarrollado en Spring Boot (Java/Kotlin), expone una API REST para operaciones administrativas, consultas públicas y analíticas. Gestiona autenticación, autorización y lógica de negocio.
  \item \textbf{Frontend}: implementado como una Single Page Application (SPA) en React + TypeScript, con vistas diferenciadas para administración, docentes y público general.
  \item \textbf{Base de datos}: basada en PostgreSQL, con un esquema normalizado que modela proyectos, personas (estudiantes y profesores), carreras, dominios de aplicación y etiquetas temáticas, entre otras entidades.
\end{itemize}

La infraestructura se despliega en contenedores, lo que permite reproducir entornos y simplifica la operación del sistema.

\subsection{Dise\~no}
El diseño del sistema se estructura en torno a varios ejes:

\paragraph{Modelo de datos.}
Se definió un modelo relacional que representa explícitamente las relaciones entre proyectos, directores, co-directores, estudiantes, carreras y dominios de aplicación. Se utilizan identificadores enteros para las claves primarias internas y UUIDs para la exposición externa de entidades, favoreciendo la estabilidad y la seguridad de las referencias.

\paragraph{Servicios de negocio.}
La lógica central del sistema se encapsula en servicios que abstraen el acceso a la base de datos y coordinan operaciones complejas (por ejemplo, importación de lotes de proyectos, generación de vistas agregadas para analíticas y gestión de respaldos).

\paragraph{Interfaz de usuario.}
El frontend aprovecha componentes reutilizables (tablas, formularios, filtros) y librerías de visualización para construir:

\begin{itemize}
  \item Tablas de administración con búsqueda, filtrado y paginación.
  \item Vistas específicas para profesores, listando sólo los proyectos asociados a cada uno.
  \item Dashboards públicos que muestran líneas de tiempo, mapas de calor temáticos, redes de colaboración entre docentes y estadísticas por carrera.
\end{itemize}

[TODO: Incluir diagramas de clases, diagramas ER y capturas de pantalla de las interfaces.]

\subsection{Comparaci\'on con objetivos}
En términos generales, la implementación alcanzada se alinea con los objetivos planteados:

\begin{itemize}
  \item Se logró centralizar la información de Trabajos Finales y Tesis en una única plataforma.
  \item Se desarrollaron interfaces diferenciadas para administración, docentes y público.
  \item Se incorporaron visualizaciones interactivas que permiten explorar la producción académica desde distintas perspectivas.
\end{itemize}

No obstante, aún existen aspectos susceptibles de mejora, particularmente en lo relativo a la optimización de consultas analíticas, la cobertura total del histórico de datos y la incorporación de nuevos perfiles de usuario (por ejemplo, estudiantes).

[TODO: Completar con referencias explícitas a resultados medibles u observaciones de uso.]

% --- 6. Usos de la app ---
\section{Usos de la app}
La plataforma \emph{ThesisFlow} se utiliza en distintos escenarios por diferentes tipos de usuarios:

\begin{itemize}
  \item \textbf{Secretaría/Administración}: para cargar nuevos proyectos aprobados, actualizar información y generar listados.
  \item \textbf{Profesores}: para consultar y gestionar los trabajos en los que participan como directores o co-directores.
  \item \textbf{Público general}: para explorar la producción del DCIC mediante visualizaciones y filtros.
\end{itemize}

\subsection{Ejemplo X}
[TODO: Describir un caso de uso concreto, por ejemplo:
(i) el flujo completo de carga de un nuevo proyecto aprobado por el consejo;
(ii) el análisis que realiza un profesor sobre su historial de dirección;
o (iii) la exploración de temas emergentes por parte de un responsable académico.]

% --- 7. Conclusiones ---
\section{Conclusiones}
El desarrollo de \emph{ThesisFlow} permitió dotar al DCIC de una herramienta específica para la gestión y visualización de Trabajos Finales y Tesis de Grado. La plataforma contribuye a:

\begin{itemize}
  \item reducir la fragmentación de la información y los procesos manuales;
  \item mejorar la accesibilidad y la transparencia de los datos académicos;
  \item habilitar nuevas formas de análisis sobre la producción del departamento.
\end{itemize}

Desde el punto de vista de ingeniería de software, el sistema constituye un caso de estudio de diseño e implementación de una aplicación web modular, con énfasis en la separación de responsabilidades, el uso de APIs REST y la integración de componentes de visualización de datos en un contexto académico real.

[TODO: Añadir conclusiones adicionales basadas en la experiencia de despliegue y uso real.]

% --- 8. Trabajo a futuro ---
\section{Trabajo a futuro}
Entre las posibles líneas de trabajo futuro se destacan:

\begin{itemize}
  \item Integrar la plataforma con otros sistemas de la Universidad (por ejemplo, sistemas de gestión académica o repositorios institucionales).
  \item Extender las funcionalidades para incluir vistas específicas para estudiantes (seguimiento del propio Trabajo Final, hitos, entrega de documentación).
  \item Optimizar el motor de analíticas, explorando consultas agregadas en base de datos, vistas materializadas o mecanismos de caché.
  \item Incorporar generación automática de reportes y estadísticas periódicas para la gestión del departamento.
  \item Mejorar el soporte para auditoría y trazabilidad (historial de cambios, registro de acciones de usuario).
\end{itemize}

[TODO: Priorizar estas líneas en función de los recursos disponibles y de la planificación institucional.]

% --- Bibliografía ---
\bibliographystyle{splncs04}
\bibliography{references}

\end{document}