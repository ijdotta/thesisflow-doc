% !TeX program = pdflatex
\documentclass[runningheads]{llncs}

% --- Packages ---
\usepackage[T1]{fontenc}
\usepackage[utf8]{inputenc}
\usepackage{graphicx}
\usepackage{hyperref}
\usepackage{booktabs}
\usepackage{csquotes}
\usepackage{amsmath}
\usepackage{enumitem}
\usepackage{listings}
\usepackage{xcolor}
\usepackage{caption}
\usepackage{subcaption}

% --- Metadata ---
\title{ThesisFlow: Plataforma Web para la Gestión y Visualización de Trabajos Finales en el DCIC}
\titlerunning{ThesisFlow}
\author{Ignacio Joaqu\'in Dotta}
\authorrunning{I. J. Dotta}
\institute{Departamento de Ciencias e Ingenier\'ia de la Computaci\'on (DCIC)\\
Universidad Nacional del Sur, Argentina\\
\email{ij.dotta@gmail.com}}

\begin{document}
\maketitle

\begin{abstract}
Este trabajo presenta \emph{ThesisFlow}, una plataforma web para la gestión y visualización de Trabajos Finales y Tesis de Grado del DCIC. El sistema integra en una única solución: (i) los flujos administrativos de carga y curado de información; (ii) un módulo de auto\-gestión para profesores; y (iii) un portal público de analíticas interactivas. Se describe la motivación y el contexto institucional, se analizan alternativas de solución, se especifican los objetivos perseguidos y se detalla la implementación a nivel de arquitectura, diseño e interfaces. A partir de la instancia desplegada se discuten usos concretos de la aplicación y se extraen conclusiones sobre su impacto en la transparencia y la toma de decisiones basada en datos. Finalmente, se identifican líneas de trabajo futuro para extender la plataforma e integrarla con otros sistemas institucionales.
\keywords{Sistemas de información académica \and Visualización de datos \and Ingeniería de software \and Aplicaciones web}
\end{abstract}

% --- 1. Motivación ---
\section{Motivaci\'on}

La gestión de Trabajos Finales y Tesis de Grado en el Departamento de Ciencias e Ingeniería de la Computación (DCIC) involucra a múltiples actores: secretaría, docentes, estudiantes y comunidad académica en general. Históricamente, la información asociada a estos proyectos se ha gestionado mediante herramientas heterogéneas (planillas, documentos aislados, repositorios parciales), lo que dificulta:

\begin{itemize}
  \item la trazabilidad del ciclo de vida de cada trabajo;
  \item el análisis de tendencias temáticas y líneas de dirección;
  \item la obtención de estadísticas confiables para la planificación académica;
  \item la transparencia institucional sobre la producción del departamento.
\end{itemize}

\subsection*{Estado actual y necesidad de sistematización}
Tradicionalmente, el DCIC no llevaba un registro centralizado ni detallado de los Trabajos Finales. Año a año, al preparar los informes de producción académica, se revisaban manualmente las actas del Consejo Departamental para reconstruir cuántos proyectos se habían aprobado. Este método es laborioso, propenso a omisiones y dificulta cualquier análisis longitudinal o temático.

En los últimos años se inició una recopilación más detallada de información (título, directores, fechas, temática, alumnos), pero aún dispersa y sin un sistema que permita explotar estos datos de forma eficiente.

\subsection*{Problema secundario: dificultades para los estudiantes}
Además de la necesidad institucional, existe una problemática estudiantil frecuente al iniciar el Trabajo Final:

\begin{itemize}
  \item Muchos estudiantes no saben qué constituye un proyecto final típico en cuanto a alcance, tipo de producto o complejidad.
  \item Otros tienen ideas iniciales, pero no saben si son viables o qué profesor trabaja en áreas relacionadas.
\end{itemize}

Una plataforma que permita visualizar proyectos históricos, sus temas y docentes involucrados facilita:

\begin{itemize}
  \item inspirarse revisando documentación de proyectos previos;
  \item entender cuáles fueron los temas más frecuentes en los últimos años;
  \item partir de un tema de interés y encontrar docentes relacionados;
  \item partir de un docente y explorar sus líneas de trabajo.
\end{itemize}

\subsection*{Síntesis}
En este contexto, \emph{ThesisFlow} surge como una herramienta integral que mejora la transparencia, reduce esfuerzos administrativos y fortalece la toma de decisiones informada tanto para estudiantes como para autoridades y docentes.

% --- 2. Exploración de soluciones ---
\section{Exploraci\'on de soluciones}

Previo al diseño del sistema se analizaron diversas plataformas existentes:

\begin{itemize}
  \item Repositorios institucionales tales como el Repositorio UBA \url{https://repositoriouba.sisbi.uba.ar/gsdl/cgi-bin/library.cgi}
  \item Repositorio Digital de la UNS \url{https://repositoriodigital.uns.edu.ar/}
  \item Plataformas de gestión académica de uso interno universitario.
  \item Herramientas de enseñanza como Moodle con módulos opcionales.
  \item Soluciones basadas en hojas de cálculo o tableros manuales.
\end{itemize}

Estos repositorios suelen cubrir únicamente la difusión del documento final, pero no ofrecen:

\begin{itemize}
  \item análisis ni visualización de datos;
  \item filtrado avanzado ni exploración temática;
  \item funcionalidades internas para la secretaría o docentes;
  \item trazabilidad en el ciclo de vida del proyecto.
\end{itemize}

Por lo tanto, si bien constituyen herramientas valiosas de archivo institucional, no abordan la problemática integral que el DCIC requiere.

% --- 3. Alternativas ---
\section{Alternativas}

Tres alternativas principales fueron consideradas:

\subsection*{1. Mantener el sistema actual}
Esta opción fue descartada por:

\begin{itemize}
  \item \textbf{Dificultad de acceso y difusión}: compartir una planilla no es equivalente a una web navegable y filtrable.
  \item \textbf{Riesgos de integridad}: las planillas no garantizan consistencia en formatos, categorías ni valores.
  \item \textbf{Nula capacidad analítica}: generar gráficos o vistas complejas requiere procesos manuales e irrepetibles.
  \item \textbf{Escalabilidad limitada}: a medida que crecen los datos, aumenta la complejidad de procesarlos manualmente.
  \item \textbf{Poca extensibilidad}: agregar nuevas funcionalidades implica rehacer la herramienta.
\end{itemize}

\subsection*{2. Adoptar una plataforma existente}
Poco viable dadas las limitaciones típicas:

\begin{itemize}
  \item modelos de datos rígidos;
  \item poca capacidad de integración con flujos administrativos locales;
  \item escaso soporte para visualizaciones interactivas.
\end{itemize}

\subsection*{3. Desarrollo propio}
Elegida como solución final, permite:

\begin{itemize}
  \item control total del diseño y reglas de negocio;
  \item optimización del análisis de datos;
  \item extensibilidad futura (nuevas carreras, posgrado, nuevos módulos);
  \item un modelo de datos adaptado al DCIC.
\end{itemize}

% --- 4. Objetivos ---
\section{Especificaci\'on de objetivos}

(Se mantiene igual — omito aquí por brevedad, pero lo mantendré íntegro en la versión final.)

% --- 5. Implementación ---
\section{Implementaci\'on}

\subsection{Arquitectura}
(Se mantiene lo que ya tenías, expandido con detalles del documento de analytics.)

\subsection{Servicios principales}

\paragraph{Servicio de Analytics.}
El \texttt{AnalyticsService} implementa:

\begin{itemize}
  \item línea de tiempo de proyectos por año y profesor,
  \item mapa de calor de temas,
  \item red de colaboración entre docentes,
  \item estadísticas por carrera,
  \item agregación por dominios de aplicación.
\end{itemize}

Cada endpoint expone datos listos para visualización, computados mediante consultas optimizadas y estructuras en memoria.

\paragraph{Autenticación y Login.}
El sistema combina dos métodos:

\begin{itemize}
  \item \textbf{Administradores}: login clásico con usuario y contraseña.
  \item \textbf{Docentes}: autenticación por \emph{magic link} enviado por email, generando un JWT temporal.
\end{itemize}

Esto elimina la necesidad de gestionar contraseñas para todos los profesores y reduce fricción.

\paragraph{Modelo de datos.}
El modelo incluye:

\begin{itemize}
  \item \texttt{Project}, \texttt{Person}, \texttt{Career}, \texttt{Tag}, \texttt{Domain}.
  \item Relaciones: un proyecto puede tener múltiples docentes y múltiples etiquetas.
  \item Uso de UUIDs para exposición pública.
\end{itemize}

(… continuaré en la versión completa …)

% --- 6. Usos ---
(Dejar igual + completar después.)

% --- 7. Conclusiones ---
(Dejar igual.)

% --- 8. Trabajo a futuro ---
\section{Trabajo a futuro}

\begin{itemize}
  \item Implementar mecanismos de \textbf{caché de vistas analíticas} para evitar recomputación costosa.
  \item Incorporar una \textbf{base NoSQL} para almacenar resultados preprocesados (filtros por docente, años, temas).
  \item Añadir \textbf{audit trails}: historial de cambios, diffs, timestamps y usuarios responsables.
  \item Escalar el sistema para incluir \textbf{trabajos de posgrado}.
  \item Extender la plataforma a \textbf{otros departamentos} de la Universidad.
  \item Permitir que los estudiantes agreguen \textbf{recursos y documentación} de sus propios trabajos.
  \item Incorporar \textbf{almacenamiento de archivos} interno para no depender solo de enlaces externos.
  \item Implementar \textbf{backups automáticos} programados.
\end{itemize}

\end{document}